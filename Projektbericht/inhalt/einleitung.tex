\section{Einleitung}
Das Themengebiet der künstlichen Intelligenz hat in den vergangenen Jahren immer mehr an Relevanz gewonnen, was sich sogar im Bereich von Rennspielen zeigt [zitat]. Zusätzlich hat die Grafikschnittstelle Open Graphics Library immer mehr Einsatz in der Computergrafik gefunden [zitat]. Aufgrund der sich aus diesen Tatsachen ergibt sich die Motivation, sowohl künstliche Intelligenz als auch die Grafikschnittstelle für ein bereits entwickeltes Rennspielframework anzuwenden und dieses System grundsätzlich zu erweitern und zu verbessern.

Zu den großen Zielen dieses Projekts zählen zum einen das Implementieren einer Künstlichen Intelligenz, welche unterschiedliche Strecken meistern kann. Außerdem soll aufgrund einer zu hohen Rechenauslastung  das bereits verwendete Grafikframework Java-Swing, worauf die Ursache der Auslastung vermutet wird, durch Open-Graphics-Library (OpenGL) ersetzt und auf dessen Leistung analysiert werden. Zusätzlich soll ein Programm entwickelt werden, das dem Nutzer das Erstellen und Speichern von Spielfeldern für das Rennspiel erleichtern soll. Des Weiteren soll das Projekt auf Apache Maven umgestellt werden, um das Erstellen und Ausführen der Applikation zu erleichtern.

Bevor der eigentliche Teil des Projekts beginnt, wird auf bereits veröffentlichte Arbeiten eingegangen, die eine ähnliche Motivation zu diesem Projekt besitzen.

\section{Related Work}
Related Work (Paper, die bereits KI für Rennspiele angewendet haben? Paper, die bereits mit OpenGL experimentiert haben?)

Im Rahmen dieser Arbeit wird zunächst der Ausgangszustand des Rennspielframeworks erläutert. Anschließend werden grundlegende Fragen geklärt, was unter künstlicher Intelligenz und Apache Maven zu verstehen ist. Des Weiteren werden verwendete Architektur-, sowie die eingesetzten Entwurfsmuster erläutert. Anschließend werden im praktischen Teil die Implementierungen der bereits beschriebenen Ziele, sowie weitere Komponenten, die zur Verbesserung des Frameworks beitragen, und die Umsetzung der KI-Algorithmen beschrieben und diskutiert. Abschließend wird ein Fazit zu diesem Projekt gezogen und ein Ausblick für zukünftige Weiterentwicklungen und mögliche Verbesserungen des Rennspielframeworks gegeben.